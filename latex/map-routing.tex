\documentclass[12pt,a4paper]{ctexart}
\usepackage{amsmath,geometry,listings}
\usepackage{xcolor}
\usepackage[linesnumbered,boxed,ruled,commentsnumbered]{algorithm2e}
\usepackage{graphicx}
\usepackage{float}
\lstset{
    numbers=left,
    keywordstyle = \color{blue},
    frame=shadowbox,
    commentstyle=\color{gray},
    identifierstyle=\color{red},
    language=c,
}
\pagestyle{plain}% 去除页眉
\geometry{left=2.54cm,right=2.54cm,top=2.5cm,bottom=2.5cm}
\fontsize{12pt}{\baselineskip}
\title{地图路由实验报告}
\author{Ruitian Zhong}
\date{2023-12-10}
\begin{document}
\maketitle
\section{实验内容}
\paragraph{背景介绍}


实验要求实现经典的 Dijkstra 最短路径算法,并对其进行优化。 这种算法广泛应用于地理信息系统(GIS),包括
MapQuest 和基于 GPS 的汽车导航系统。
地图。本次实验对象是图 maps 或 graphs,其中顶点为平面上的点,这些点由权值为欧氏距离的边相连成图。
可将顶点视为城市,将边视为相连的道路。在文件示例中表示地图,首先列出了其中顶点数和边数,然后
列出顶点(索引后跟其 x 和 y 坐标),然后列出边(顶点对),最后列出源点和汇点。 例如,如下左图信
息表示右图:
\paragraph{目标}
优化 Dijkstra 算法,使其可以处理给定图的数万条最短路径查询。 一旦你读取图(并可选地预处
理),你的程序应该在亚线性时间内解决最短路径问题。

\section{实验算法和实现}


\subsection{处理输入数据}

\paragraph{}
实验要求从一组格式规范的输入当中获得点、边的信息,为实现这个功能,
专门实现了\textit{ReadGraph.java},这个类中的\textit{readGraph()}静态方法将文件中
的数据实例化为一个类,方便上层的实现。

\subsection{算法概述}
对于寻找给定源点和汇点,寻找两点之间的最短路径的问题,
采用经典的 Dijkstra算法。在Dijkstra算法的计算过程当中,
需要寻找当前与源点距离最近的点(还没有被选中),为减少搜索
的次数,采取优先队列(最小)堆的方法,该方法插入的时间复杂度为:

\begin{equation}
    \nonumber
    \mathcal{O}{(\log(n))}
\end{equation}

该方法删除最小元素的时间复杂度为
\begin{equation}
    \nonumber
    \mathcal{O} {(1)}
\end{equation}


因此该方法相比简单遍历整个距离数组寻找最大值,可以有效降低执行大量最短路径时计算复杂度的开销,为后续的优化
打下了基础

\subsection{算法优化}

\paragraph{找到给定汇点就退出} 经典的Dijkstra算法找到所有节点到
源点的最短距离才退出,由于查询的时候给定了源点和终点,当终点的最短距离
被找到的时候,无需计算其他节点退出可节省一定时间。

\paragraph{快速初始化距离数组}
一般的实现会在开始Dijkstra算法开始前初始化距离数组和前驱节点数组。
由于有8万多个节点和大量的查询,因此,初始化的开销也不能忽视。因此,
每次将距离数组的值从无穷大改为非无穷大的时候,记录该数组的下标,下一次
查询前,只初始化改变过的位置,以期减少初始化的开销,代码如下:

\begin{lstlisting}[basicstyle=\ttfamily,caption=map-routing/Dijkstra.java,language=java,showstringspaces=false,firstnumber=105]
    private void fastReinit() {
        while (top != -1) {
            int i = modify[top--];
            distTo[i] = Double.POSITIVE_INFINITY;
            edgeTo[i] = null;
        }
    }

    private void slowReinit() {
        for (int i = 0; i < graph.vertex(); i++) {
            distTo[i] = Double.POSITIVE_INFINITY;
            edgeTo[i] = null;
        }

    }
\end{lstlisting}

\subsection{实验实现}
在实验给定的输入文件的基础上进行以下测试,评估算法优化的效果
\begin{enumerate}
    \item 基于原始的Dijkstra算法,不进行优化,随机生成1000个源点-终点对,查找它们之间的最短路径
    \item 基于原始的Dijkstra算法,找到给定汇点就退出,随机生成1000个源点-终点对,查找它们之间的最短路径
    \item 基于原始的Dijkstra算法,找到给定汇点就退出,同时进行上文所述的快速初始化,查找他们之间的最短路径
          分别对上述3个实验进行计时比较它们的结果
\end{enumerate}



\section{实验结果和分析}

\subsection{实验结果}

下面的表格给出三个实验的耗时
\begin{table}[htbp]
    \centering
    \begin{tabular}{|l|l|}
        \hline
        实验编号 & 时间(ms) \\ \hline
        1        & 20908      \\ \hline
        2        & 11321      \\ \hline
        3        & 11294      \\ \hline
    \end{tabular}
\end{table}

\subsection{结果分析}

\paragraph{找到给定汇点就退出的方法提高了搜索速度}
可以看到找到给定汇点最短距离就退出的方法的耗时为不做
任何优化的 $54\%$,较大幅度地降低了搜索开销。

\paragraph{快速初始化数组有一定优化作用,但幅度不大}
在实验2的基础上,可以看到快速初始化数组时间略快,但幅度不是非常大,
分析后发现,随机生成的源点-终点对使得需要被重新初始化的节点数量仍然
很大,这可能是其优化作用有限的原因



\section{更多可能的优化}
\paragraph{$A^*$ 算法}
$A^*$算法是一种启发式算法,在保证正确性的前提下可以加快搜索速度,进而加快查询的
速度。

\paragraph{合理地缓存已经计算过的结果}
在本次实验每次查询都复用了一个Dijkstra对象中的数据结构,上一次
计算的结果会被清除。想象有这样一种情况:连续两次搜索的源点相同,如果能够
利用上一次查询的时候计算的结果,那将会加快第二次搜索的速度,如果实际中用户
多次连续搜索指定同一个源点(可能可以称之为查询的局部性),那么这种优化方法
的效果可能会更优。


此外,由于存储所有源点-终点对的计算结果开销太大,可能可以合理缓存一定数量的
计算结果(比如说100个),这种优化对于多个自身有一定查询局部性的用户的查询
可能有一定效果。如果要实现这种算法还需要实现计算结果缓存的逐出(\textit{eviction})算法,
比如说最近最少使用算法(\textit{LRU})等。

\section{收获}
这次实验在课内学习到的\textit{Dijkstra}算法的基础上,根据地图路由问题的实际情况做出一系列的优化,并通过性能评测有效评估算法优化的效果和最初假设的正确性,提高了解决实际问题
的能力。

\end{document}
\documentclass[12pt,a4paper]{ctexart}
\usepackage{amsmath,geometry,listings}
\usepackage{xcolor}
\usepackage[linesnumbered,boxed,ruled,commentsnumbered]{algorithm2e}
\usepackage{graphicx}

\usepackage{float}
\lstset{
    numbers=left,
    keywordstyle = \color{blue},
    frame=shadowbox,
    commentstyle=\color{gray},
    identifierstyle=\color{red},
    language=c,
}
\pagestyle{plain}% 去除页眉
\geometry{left=2.54cm,right=2.54cm,top=2.5cm,bottom=2.5cm}
\fontsize{12pt}{\baselineskip}
\title{排序算法性能比较实验报告}
\author{Ruitian Zhong}
\date{2023-10-24}
\begin{document}
\maketitle
\section{实验内容}
\paragraph{}实现插入排序(Insertion Sort,IS),自顶向下归并排序(Top-down Mergesort,TDM),自底向上归并排
序(Bottom-up Mergesort,BUM),随机快速排序(Random Quicksort,RQ),Dijkstra 3-路划分快速排序
(Quicksort with Dijkstra 3-way Partition,QD3P)。在你的计算机上针对不同输入规模数据进行实验,对
比上述排序算法的时间及空间占用性能。要求对于每次输入运行 10 次,记录每次时间/空间占用,取平
均值。

\section{实验方法}
\subsection{时间计算}
\paragraph{}在使用每种排序算法排序前记录下开始时间$start$(以纳秒为单位),调用结束后
记录下结束时间$end$,单次排序消耗的时间为
$$ T = end - start $$

\begin{lstlisting}[basicstyle=\ttfamily,caption=sort/test.java,language=java,showstringspaces=false,firstnumber=140]
    for (int j = 0; j < 5; j++) {
        int[] ephemeral = arr.clone();
        start = System.nanoTime();
        s[j].sort(ephemeral);
        end = System.nanoTime();
        temporal[j][i] = end - start;
        memory[j][i] = s[j].memory();
    }
\end{lstlisting}



\subsection{空间占用计算}
\paragraph{}为了计算最大空间占用,实现了\textit{MemoryMonitor}类,该
类主要提供了两个函数调用,一个是\textit{void update(int x)},其中x为使用内存的变化量,
当使用内存时$x>0$,释放内存时$x<0$,这样能够大致统计递归函数调用时的最大内存用量;
\textit{void getMaxMemory()}获取排序时消耗的最大内存。
\lstinputlisting[language=java,basicstyle=\ttfamily,caption=sort/MemoryMonitor.java]{../sort/MemoryMonitor.java}

\subsection{实验流程}
\begin{enumerate}
    \item 规模为10000,随机生成的数组排序
    \item 规模为50000,随机生成的数组排序
    \item 规模为100000,随机生成的数组排序
    \item 规模为10000,生成降序数组排序(将降序数组排序成升序数组)
    \item 规模为10000,所有元素相同的数组排序
\end{enumerate}


\section{实验结果}

\begin{table}[H]
    \setlength{\abovecaptionskip}{0cm}
    \setlength{\belowcaptionskip}{0.5cm}
    \small
    \centering
    \caption[short]{规模为10000,随机生成的数组排序,时间/ns}

    \begin{tabular}{|l|l|l|l|l|l|}
        \hline
                & IS       & TDM     & BUM     & RQ      & QD3P    \\ \hline
        1       & 18747700 & 2514500 & 2727400 & 2230300 & 2505000 \\ \hline
        2       & 35722300 & 723400  & 1058000 & 478200  & 418300  \\ \hline
        3       & 13312000 & 961800  & 1166900 & 468200  & 770400  \\ \hline
        4       & 13149500 & 821800  & 1069000 & 492700  & 320600  \\ \hline
        5       & 12987700 & 801200  & 1131300 & 509000  & 295500  \\ \hline
        6       & 12937700 & 877600  & 1097000 & 449100  & 308600  \\ \hline
        7       & 12940100 & 828200  & 1121600 & 458100  & 306000  \\ \hline
        8       & 13276600 & 822400  & 714800  & 508900  & 305600  \\ \hline
        9       & 13139400 & 841500  & 703400  & 467900  & 303300  \\ \hline
        10      & 12783700 & 923700  & 772600  & 497000  & 300100  \\ \hline
        Average & 15899670 & 1011610 & 1156200 & 655940  & 583340  \\ \hline
    \end{tabular}

\end{table}

\begin{table}[H]
    \setlength{\abovecaptionskip}{0cm}
    \setlength{\belowcaptionskip}{0.5cm}
    \small
    \centering
    \caption[short]{规模为10000,随机生成的数组排序,内存/kB}
    \begin{tabular}{|l|l|l|l|l|l|}
        \hline
                & IS    & TDM    & BUM    & RQ     & QD3P   \\ \hline
        1       & 0.012 & 39.129 & 39.082 & 0.43   & 0.203  \\ \hline
        2       & 0.012 & 39.129 & 39.082 & 0.488  & 0.188  \\ \hline
        3       & 0.012 & 39.129 & 39.082 & 0.43   & 0.234  \\ \hline
        4       & 0.012 & 39.129 & 39.082 & 0.41   & 0.188  \\ \hline
        5       & 0.012 & 39.129 & 39.082 & 0.41   & 0.219  \\ \hline
        6       & 0.012 & 39.129 & 39.082 & 0.527  & 0.219  \\ \hline
        7       & 0.012 & 39.129 & 39.082 & 0.43   & 0.188  \\ \hline
        8       & 0.012 & 39.129 & 39.082 & 0.449  & 0.172  \\ \hline
        9       & 0.012 & 39.129 & 39.082 & 0.449  & 0.25   \\ \hline
        10      & 0.012 & 39.129 & 39.082 & 0.43   & 0.188  \\ \hline
        Average & 0.012 & 39.129 & 39.082 & 0.4453 & 0.2049 \\ \hline
    \end{tabular}
\end{table}





\begin{table}[H]
    \setlength{\abovecaptionskip}{0cm}
    \setlength{\belowcaptionskip}{0.5cm}
    \small
    \centering
    \caption[short]{规模为50000,随机生成的数组排序,时间/ns}
    \begin{tabular}{|l|l|l|l|l|l|}
        \hline
                & IS        & TDM     & BUM     & RQ      & QD3P    \\ \hline
        1       & 347772700 & 4226400 & 3537800 & 2342500 & 1809000 \\ \hline
        2       & 330399600 & 3359800 & 3137600 & 2419800 & 1611900 \\ \hline
        3       & 321177400 & 4301400 & 3385700 & 2266000 & 1478000 \\ \hline
        4       & 385847900 & 4790600 & 3535000 & 2785100 & 1741600 \\ \hline
        5       & 352844700 & 4694800 & 3709400 & 2450100 & 1533900 \\ \hline
        6       & 337261100 & 3993200 & 2958500 & 2184300 & 1462500 \\ \hline
        7       & 350169200 & 5381900 & 3861900 & 2576900 & 1661800 \\ \hline
        8       & 351262500 & 4218400 & 3433700 & 2312400 & 1490800 \\ \hline
        9       & 324163000 & 4294900 & 3336500 & 2288600 & 1490100 \\ \hline
        10      & 339655600 & 4633800 & 3466800 & 2370100 & 1531800 \\ \hline
        Average & 344055370 & 4389520 & 3436290 & 2399580 & 1581140 \\ \hline
    \end{tabular}
\end{table}

\begin{table}[H]
    \setlength{\abovecaptionskip}{0cm}
    \setlength{\belowcaptionskip}{0.5cm}
    \small
    \centering
    \caption[short]{规模为50000,随机生成的数组排序,内存/kB}
    \begin{tabular}{|l|l|l|l|l|l|}
        \hline
                & IS    & TDM     & BUM     & RQ     & QD3P   \\ \hline
        1       & 0.012 & 195.387 & 195.332 & 0.566  & 0.234  \\ \hline
        2       & 0.012 & 195.387 & 195.332 & 0.508  & 0.172  \\ \hline
        3       & 0.012 & 195.387 & 195.332 & 0.508  & 0.203  \\ \hline
        4       & 0.012 & 195.387 & 195.332 & 0.449  & 0.203  \\ \hline
        5       & 0.012 & 195.387 & 195.332 & 0.508  & 0.203  \\ \hline
        6       & 0.012 & 195.387 & 195.332 & 0.547  & 0.188  \\ \hline
        7       & 0.012 & 195.387 & 195.332 & 0.469  & 0.188  \\ \hline
        8       & 0.012 & 195.387 & 195.332 & 0.488  & 0.203  \\ \hline
        9       & 0.012 & 195.387 & 195.332 & 0.527  & 0.234  \\ \hline
        10      & 0.012 & 195.387 & 195.332 & 0.488  & 0.203  \\ \hline
        Average & 0.012 & 195.387 & 195.332 & 0.5058 & 0.2031 \\ \hline
    \end{tabular}
\end{table}






\begin{table}[H]
    \setlength{\abovecaptionskip}{0cm}
    \setlength{\belowcaptionskip}{0.5cm}
    \small
    \centering
    \caption[short]{规模为100000,随机生成的数组排序,时间/ns}
    \begin{tabular}{|l|l|l|l|l|l|}
        \hline
                & IS         & TDM      & BUM     & RQ      & QD3P    \\ \hline
        1       & 1412390600 & 8269800  & 7583300 & 4970800 & 3995400 \\ \hline
        2       & 1373270300 & 8632700  & 6964500 & 4724600 & 3182100 \\ \hline
        3       & 1453717400 & 10812500 & 7290300 & 4809700 & 3312200 \\ \hline
        4       & 1325760700 & 8636500  & 6687500 & 4569100 & 3113900 \\ \hline
        5       & 1267842600 & 8448200  & 6868900 & 4603000 & 2979400 \\ \hline
        6       & 1284115500 & 8491100  & 6805500 & 4636700 & 2883400 \\ \hline
        7       & 1302477200 & 8607900  & 6809700 & 4703100 & 2845600 \\ \hline
        8       & 1289560200 & 8569400  & 7073300 & 4636800 & 2993500 \\ \hline
        9       & 1275855800 & 8582500  & 6814100 & 4773400 & 2962900 \\ \hline
        10      & 1334560100 & 9216300  & 6856100 & 4726500 & 2833800 \\ \hline
        Average & 1331955040 & 8826690  & 6975320 & 4715370 & 3110220 \\ \hline
    \end{tabular}

\end{table}




\begin{table}[H]
    \setlength{\abovecaptionskip}{0cm}
    \setlength{\belowcaptionskip}{0.5cm}
    \small
    \centering
    \caption[short]{规模为100000,随机生成的数组排序,内存/kB}
    \begin{tabular}{|l|l|l|l|l|l|}
        \hline
                & IS    & TDM     & BUM     & RQ     & QD3P  \\ \hline
        1       & 0.012 & 390.703 & 390.645 & 0.508  & 0.188 \\ \hline
        2       & 0.012 & 390.703 & 390.645 & 0.508  & 0.234 \\ \hline
        3       & 0.012 & 390.703 & 390.645 & 0.488  & 0.188 \\ \hline
        4       & 0.012 & 390.703 & 390.645 & 0.508  & 0.203 \\ \hline
        5       & 0.012 & 390.703 & 390.645 & 0.508  & 0.203 \\ \hline
        6       & 0.012 & 390.703 & 390.645 & 0.586  & 0.188 \\ \hline
        7       & 0.012 & 390.703 & 390.645 & 0.547  & 0.172 \\ \hline
        8       & 0.012 & 390.703 & 390.645 & 0.586  & 0.172 \\ \hline
        9       & 0.012 & 390.703 & 390.645 & 0.547  & 0.203 \\ \hline
        10      & 0.012 & 390.703 & 390.645 & 0.547  & 0.219 \\ \hline
        Average & 0.012 & 390.703 & 390.645 & 0.5333 & 0.197 \\ \hline
    \end{tabular}
\end{table}



\begin{table}[H]
    \setlength{\abovecaptionskip}{0cm}
    \setlength{\belowcaptionskip}{0.5cm}
    \small
    \centering
    \caption[short]{规模为10000,生成降序数组排序,时间/ns}
    \begin{tabular}{|l|l|l|l|l|l|}
        \hline
                & IS       & TDM     & BUM    & RQ     & QD3P      \\ \hline
        1       & 27334300 & 413300  & 282500 & 446800 & 109686500 \\ \hline
        2       & 25383300 & 423400  & 292600 & 337000 & 108032800 \\ \hline
        3       & 25422700 & 447400  & 298700 & 329000 & 124322300 \\ \hline
        4       & 28759500 & 460000  & 316600 & 373900 & 116187200 \\ \hline
        5       & 25298100 & 425800  & 313300 & 358200 & 108637400 \\ \hline
        6       & 25334800 & 464400  & 339800 & 336100 & 108694000 \\ \hline
        7       & 25334400 & 2208700 & 196400 & 306800 & 105843000 \\ \hline
        8       & 26227000 & 235600  & 196600 & 290400 & 106899000 \\ \hline
        9       & 25460900 & 412500  & 283100 & 345100 & 110229200 \\ \hline
        10      & 25998200 & 238700  & 215200 & 301500 & 107254600 \\ \hline
        Average & 26055320 & 572980  & 273480 & 342480 & 110578600 \\ \hline
    \end{tabular}
\end{table}

\begin{table}[H]
    \setlength{\abovecaptionskip}{0cm}
    \setlength{\belowcaptionskip}{0.5cm}
    \small
    \centering
    \caption[short]{规模为10000,生成降序数组排序,内存/kB}
    \begin{tabular}{|l|l|l|l|l|l|}
        \hline
                & IS    & TDM    & BUM    & RQ     & QD3P    \\ \hline
        1       & 0.012 & 39.129 & 39.082 & 0.566  & 156.234 \\ \hline
        2       & 0.012 & 39.129 & 39.082 & 0.605  & 156.234 \\ \hline
        3       & 0.012 & 39.129 & 39.082 & 0.605  & 156.234 \\ \hline
        4       & 0.012 & 39.129 & 39.082 & 0.527  & 156.234 \\ \hline
        5       & 0.012 & 39.129 & 39.082 & 0.625  & 156.234 \\ \hline
        6       & 0.012 & 39.129 & 39.082 & 0.547  & 156.234 \\ \hline
        7       & 0.012 & 39.129 & 39.082 & 0.586  & 156.234 \\ \hline
        8       & 0.012 & 39.129 & 39.082 & 0.605  & 156.234 \\ \hline
        9       & 0.012 & 39.129 & 39.082 & 0.703  & 156.234 \\ \hline
        10      & 0.012 & 39.129 & 39.082 & 0.645  & 156.234 \\ \hline
        Average & 0.012 & 39.129 & 39.082 & 0.6014 & 156.234 \\ \hline
    \end{tabular}
\end{table}


\begin{table}[H]
    \setlength{\abovecaptionskip}{0cm}
    \setlength{\belowcaptionskip}{0.5cm}
    \small
    \centering
    \caption[short]{规模为10000,所有元素相同的数组排序,时间/ns}

    \begin{tabular}{|l|l|l|l|l|l|}
        \hline
                & IS     & TDM     & BUM     & RQ      & QD3P   \\ \hline
        1       & 847000 & 1363500 & 1511500 & 1391600 & 173500 \\ \hline
        2       & 199700 & 897400  & 1207900 & 457200  & 162800 \\ \hline
        3       & 178900 & 778900  & 1190800 & 505500  & 184200 \\ \hline
        4       & 225700 & 5105500 & 717100  & 435000  & 168000 \\ \hline
        5       & 197400 & 3198100 & 734000  & 403700  & 181600 \\ \hline
        6       & 217800 & 398500  & 692500  & 398700  & 167500 \\ \hline
        7       & 204900 & 393600  & 560000  & 401300  & 199500 \\ \hline
        8       & 55100  & 381900  & 264600  & 397000  & 56200  \\ \hline
        9       & 34500  & 398100  & 260000  & 418400  & 35300  \\ \hline
        10      & 29500  & 388700  & 258800  & 408700  & 24900  \\ \hline
        Average & 219050 & 1330420 & 739720  & 521710  & 135350 \\ \hline
    \end{tabular}
\end{table}


\begin{table}[H]
    \setlength{\abovecaptionskip}{0cm}
    \setlength{\belowcaptionskip}{0.5cm}
    \small
    \centering
    \caption[short]{规模为10000,所有元素相同的数组排序,内存/kB}
    \begin{tabular}{|l|l|l|l|l|l|}
        \hline
                & IS    & TDM    & BUM    & RQ    & QD3P  \\ \hline
        1       & 0.012 & 39.129 & 39.082 & 0.254 & 0.016 \\ \hline
        2       & 0.012 & 39.129 & 39.082 & 0.254 & 0.016 \\ \hline
        3       & 0.012 & 39.129 & 39.082 & 0.254 & 0.016 \\ \hline
        4       & 0.012 & 39.129 & 39.082 & 0.254 & 0.016 \\ \hline
        5       & 0.012 & 39.129 & 39.082 & 0.254 & 0.016 \\ \hline
        6       & 0.012 & 39.129 & 39.082 & 0.254 & 0.016 \\ \hline
        7       & 0.012 & 39.129 & 39.082 & 0.254 & 0.016 \\ \hline
        8       & 0.012 & 39.129 & 39.082 & 0.254 & 0.016 \\ \hline
        9       & 0.012 & 39.129 & 39.082 & 0.254 & 0.016 \\ \hline
        10      & 0.012 & 39.129 & 39.082 & 0.254 & 0.016 \\ \hline
        Average & 0.012 & 39.129 & 39.082 & 0.254 & 0.016 \\ \hline
    \end{tabular}
\end{table}



\section{实验结果分析与回答问题}
\begin{enumerate}
    \item Which sort worked best on data in constant or increasing order (i.e., already sorted data)? Why do you think
          this sort worked best?
          \paragraph{回答}
          插入排序在数组为常数或升序的时候性能最好,因为在有序的情况下插入排序的内层循环只执行一次,时间复杂度在这种情况下降为$\mathcal{O}(n)$,
          空间复杂度为$mathcal{O}(1)$
    \item Did the same sort do well on the case of mostly sorted data? Why or why not?
          \paragraph{回答}
          不是同一个排序算法在所有情况下都做得很好。比如说插入排序在数组有序下性能好,但数组随机生成的时候
          数组下降得很快,又比如,归并排序在数据量较小得情况下,效率不如插入排序高。

    \item In general, did the ordering of the incoming data affect the performance of the sorting algorithms? Please
          answer this question by referencing specific data from your table to support your answer.

          \paragraph{回答} 会有影响。比如说有10000个数据,这10000个数据在$[0,100)$范围内的时候,三路切分比快速排序的速度要快($3110220<4715370$)


    \item  Which sort did best on the shorter (i.e., n = 1,000) data sets? Did the same one do better on the longer (i.e.,
          n = 100,000) data sets? Why or why not? Please use specific data from your table to support your answer.

          \paragraph{回答}
          数据量比较小($n=100$)的时候插入排序表现较好,随着数据的增大,QD3P的表现变好,主要是数据量变多,重复的数据增加,
          三路切分处理重复的情况效率比较高,因此速度最快。
          
    \item In general, which sort did better? Give a hypothesis as to why the difference in performance exists.

          \paragraph{回答}
          \textit{QD3P}做得比较好,它的时间复杂度为$\mathcal{O}(\lg{n})$,而且能够处理好重复数据较多的情况(实验证明了它在这种情况下是最快的)

    \item Are there results in your table that seem to be inconsistent? (e.g., If I get run times for a sort that look like
          this [1.3, 1.5, 1.6, 7.0, 1.2, 1.6, 1.4, 1.8, 2.0, 1.5] the 7.0 entry is not consistent with the rest). Why do you think
          this happened

          \paragraph{回答} 确实有一些异常数据出现,我认为是由于生成的随机数据导致了某种比较差的初始情况,此外,
          \textit{Java Virtual Machine(JVM)}会定时进行垃圾收集,导致一定程度上的延迟。

\end{enumerate}


\end{document}




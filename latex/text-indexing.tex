\documentclass[12pt,a4paper]{ctexart}
\usepackage{amsmath,geometry,listings}
\usepackage{xcolor}
\usepackage[linesnumbered,boxed,ruled,commentsnumbered]{algorithm2e}
\usepackage{graphicx}
\usepackage{float}
\lstset{
    numbers=left,
    keywordstyle = \color{blue},
    frame=shadowbox,
    commentstyle=\color{gray},
    identifierstyle=\color{red},
    language=c,
}
\pagestyle{plain}% 去除页眉
\geometry{left=2.54cm,right=2.54cm,top=2.5cm,bottom=2.5cm}
\fontsize{12pt}{\baselineskip}
\title{文本索引实验报告}
\author{Ruitian Zhong}
\date{2023-12-16}
\begin{document}
\maketitle
\section{实验内容}
\paragraph{背景介绍}

程序应该读取语料库,将其存储为(可能巨大)字符串,并可能为其创建索引,如下所述。 然后
它应该逐个读取查询(假设在命令行中的第二个命名文件中,每行有一个查询),并打印出语料库中每个查
询在文本文件中首次出现的位置。

\section{实验算法和实现}

\subsection{三种可能的实现方式}
第一种解决办法就是暴力搜索(Brute Force),使用该方法搜索
一次的时间复杂度为:

\begin{equation}
    \nonumber
    \mathcal{O}{(n * m)}
\end{equation}

其中$m$是模式字符串的长度,$n$是主串的长度

\begin{lstlisting}[basicstyle=\ttfamily,caption=text-indexing/BruteForceSearch.java,language=java,showstringspaces=false,firstnumber=105]

@Override
public int search(String pattern) {
int i = 0, j = 0;
while (j < content.length()) {
for (i = 0; 
     i < pattern.length() && i + j < content.length(); 
     i++) {
if (pattern.charAt(i) != content.charAt(j + i)) {
    break;
}
}
     if (i == pattern.length()) {
                return j;
        }
            j++;
        }
        return -1;
    }

\end{lstlisting}

第二种解决方法使用KMP匹配算法,其时间复杂度为
\begin{equation}
    \nonumber
    \mathcal{O}{(n + m)}
\end{equation}
其中$n$为原字符串的长度,$m$为模式字符串的长度。

\begin{lstlisting}[basicstyle=\ttfamily,caption=text-indexing/KMPSearch.java.java,language=java,showstringspaces=false]
private int[] computeNext(String pattern) {
    int[] next = new int[pattern.length()];
    int j = 0, k = -1;
    next[0] = -1;
    while (j < pattern.length() - 1) {
   if (k == -1 || pattern.charAt(k) == pattern.charAt(j)) {
            j++;
            k++;
            next[j] = k;
        } else {
            k = next[k];
        }
    }
    return next;
}

@Override
public int search(String pattern) {
    int i = 0, j = 0;
    int index = -1;
    int[] next = computeNext(pattern);
    while (i < pattern.length() && j < content.length()) {
    if (i == -1 || pattern.charAt(i) == content.charAt(j)) {
            i++;
            j++;
        } else {
            i = next[i];
        }
    }
    if (i == pattern.length()) {
        index = j - pattern.length();
    }
    return index;
}
\end{lstlisting}


第三种解决办法是建立索引,加速查找的速度(建立索引需要时间)。
我选择了使用后缀数组的方法,简单来说,构建一个指向字符串每个单词开始
位置的指针/索引数组,然后根据字典序对该数组进行排序,搜索的时候使用
二分查找的办法,可以将搜索相应字符串时算法的时间复杂度降低到对数级别(在模式串比较短
的情况下),在这种方法下使用单词为分割边界,原因是减少建立索引时的时间、空间开销。

因此,设计了TextSearch接口:
\begin{lstlisting}[basicstyle=\ttfamily,caption=text-indexing/TextSearch.java.java,language=java,showstringspaces=false]

public interface TextSearch {
    int search(String pattern);
}
\end{lstlisting}

BruteForceSearch、KMPSearch
和IndexedSearch
三个类分别实现\textit{TextSearch}这个接口。

\subsection{测试文件生成}

为了充分比较以上三种文本搜索方法的效率,不可能人工手动搜索,
因此采用程序生成测试文件
\begin{lstlisting}[basicstyle=\ttfamily,language=java,showstringspaces=false]

public static void generateQuery() throws IOException {
    File file = new File("query.txt");
    if (!file.exists()) {
        file.createNewFile();
    }
     FileWriter fileWriter = 
                   new FileWriter(file.getName(), false);
    var lines = getAllLine();
    Random r = new Random(System.currentTimeMillis());
    for (int i = 0; i < 6000; i++) {
  int lineNum = r.nextInt(lines.size());
  String line = lines.get(lineNum);
  String sub = line.substring(r.nextInt(line.length()));
  fileWriter.write(sub + "\n");
 }
    for (int i = 0; i < 3000; i++) {
        int lineNum = r.nextInt(lines.size());
        String line = lines.get(lineNum);
String sub = line.substring(r.nextInt(line.length()));
        if (sub.length() > 10) {
            sub = sub.substring(0, 10);
        }
        fileWriter.write(sub + "\n");
    }
    for (int i = 0; i < 3000; i++) {
    int length = r.nextInt(10) + 1;
    fileWriter.write(generateRandomString(length) + "\n");
    }
    for (int i = 0; i < 1000; i++) {
        fileWriter.write(UUID.randomUUID() + "\n");
    }

    fileWriter.close();

}
\end{lstlisting}

简单来说,生成6000个原文中就有的字符串(随机截取),随机产生3000个长度
不超过11的字符串,随机产生1000个长的随字符串(也就是UUID)




\section{实验结果和分析}

\subsection{实验设置}
分别使用BruteForceSearch、KMPSearch、IndexedSearch
执行程序生成的查询字符串(上文已说明如何产生),对
同一个查询字符串文件,分别测试三种程序完成所有查询消耗的时间,并进行
比较。

\subsection{实验结果}

下面的表格给出三个实验的耗时

\begin{table}[htbp]
    \centering
    \begin{tabular}{|l|l|}
        \hline
        实验编号          & 时间(ms) \\ \hline
        BruteForce Search & 8066       \\ \hline
        KMP Search        & 6631       \\ \hline
        Indexed Search    & 4133       \\ \hline
    \end{tabular}
\end{table}
\subsection{结果分析}

可以看到,从搜索效率上来看有:

\begin{equation}
    \nonumber
    IndexedSearch > KMPSearch > BruteForce Search
\end{equation}

但也要关注到,IndexedSearch建立索引的时间较长,如果只是数量有限
的查询,应该优先使用\textit{KMP}算法进行搜索。


\section{收获}
本次实验通过实现三种文本搜索并进行性能评估,深入了解了它们
之间的性能差距,并了解到它们适用的使用场景。

\end{document}